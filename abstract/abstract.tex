\documentclass{article}

\begin{document}
	\pagestyle{empty}
	\begin{center}
		\large Binghamton University \\
		\vspace*{5pt}
		\Large\textsc{Effective algebro-homotopical constructions and their applications} \\
		\vspace*{10pt}
		\normalsize Anibal M. Medina-Mardones \\
	\end{center}
	\textbf{Abstract.} It is necessary in order to incorporate ideas from homotopy theory into concrete contexts -- such as topological data analysis and topological lattice field theory -- to have effective constructions of concepts defined only indirectly or transcendentally.
	In groundbreaking work, Mandell showed that the entire homotopy type of a space was encoded in the quasi-isomorphism type of its cochains enhanced with an $E_\infty$-structure.
	In this talk, we will present a concrete construction of such structure by explicitly restoring up to coherent homotopies the broken symmetry of the diagonal of cellular spaces, and, on the way, we will explore applications of these ideas to data science, theoretical physics, knot theory, higher category theory, and convex and toric geometry.


%	This talk will discuss how to effectively construct homotopy invariants using the broken symmetry of the diagonal map of a cellular space.
%	More specifically, from an algebra structure on its cochains that is both commutative and associative up to coherent homotopies.
%	This algebraic structure encodes the entire homotopy type of the space by a result of Mandell, and during the talk it will be described in terms of a finite number of generating multi-operations.
%	Time permitting, this talk will also outline some applications of such constructions.
%	Together with allowing for the concrete computation of finer cohomological invariants in persistent homology --Steenrod barcodes-- these effective constructions also reveal combinatorial information connected to convex geometry and higher category theory.
\end{document}