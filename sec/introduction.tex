%!TEX root = ../effc_top.tex

\begin{frame}{Viewpoint}
	\vskip -10pt
	\begin{block}{A primary goal of algebraic topology}
		To construct invariants of topological space up to some notion of equivalence by recasting into combinatorial and algebraic models.
	\end{block}

	\medskip\pause
	\begin{block}{A basic tension}
		Computability vs. strength of invariants.
	\end{block}

	\medskip \textcolor{pblue}{Example:}
	Homology vs. homotopy.

	\medskip\pause
	\begin{block}{A more subtle tension}
		Effectiveness vs. functoriality of their constructions.
	\end{block}

	\medskip \textcolor{pblue}{Example:}
	cohomology via a cochain complex or \\
	\hspace*{40pt} via maps to Eilenberg-Maclane spaces.
\end{frame}

\begin{frame}{Modeling spaces combinatorially}
	\pause
	\begin{block}{Poincar\'{e}}
		Break spaces into contractible combinatorial pieces: simplices, cubes, ...
	\end{block}

	\pause \textcolor{pblue}{Cohomology:}
	via a cochain complex generated by these pieces.

	\medskip\pause	More generally:
	\begin{block}{Kan-Quillen}
		Use category theory to replace spaces by functors with a geometric realization: simplicial sets, cubical sets, ...
	\end{block}

	\pause \textcolor{pblue}{Basic objects:}
	Chains on standard pieces $\gchains(\gsimplex^n)$, $\gchains(\gcube^n)$, ...

	\smallskip\pause
	\begin{block}{Our goal (loosly stated)}
		Understand these chain complexes deeply to enhance (co)homology with finer effectively computable invariants.
	\end{block}
\end{frame}

\begin{frame}{Shortcomings of (co)homology}
	\pause With mod 2 coefficients the real projective plane and the wedge of a sphere and a circle are isomorphic
	\[
	H^\bullet(\R \rP^2; \Ftwo) \cong H^\bullet(S^1 \vee S^2; \Ftwo)
	\]
	as graded vector spaces.

	\bigskip\pause
	Similarly,
	\[
	H^\bullet(\bC \rP^2; \Z) \cong H^\bullet(S^2 \vee S^4; \Z)
	\]
	as graded abelian groups.

	\bigskip\pause
	\begin{block}{Cup product}
		These can be distinguished by the product in cohomology.
	\end{block}
\end{frame}