%!TEX root = ../effc_top.tex

\begin{frame}{Viewpoint}
	\pause
	\colorit{A goal of algebraic topology} \\
	To construct invariants of spaces up to some notion of equivalence.

	\bigskip\pause
	\colorit{Today} \\
	CW complexes and homotopy equivalence.

	\bigskip\pause
	\colorit{A basic tension} \\
	Computability \colorit{vs} strength of invariants.

	\bigskip\pause
	\colorit{Example} \\
	Cohomology \colorit{vs} homotopy.

	\bigskip\pause
	\colorit{A more subtle one} \\
	Effectiveness \colorit{vs} functoriality of their constructions.

	\bigskip\pause
	\colorit{Example} \\
	Cohomology via chain complex \colorit{vs} maps to Eilenberg-Maclane spaces.
\end{frame}

\begin{frame}{Effectively defined cohomology}
	\pause
	\colorit{Poincar\'{e}'s idea} \\
	Break spaces into contractible combinatorial pieces: \\
	\begin{center}
		Simplices, cubes, ...
	\end{center}

	\pause
	\colorit{Kan's idea} \\
	Replace spaces by functors with a geometric realization: \\
	\begin{center}
		Simplicial sets, cubical sets, ...
	\end{center}

	\pause
	\colorit{Compute cohomology} \\
	Using a chain complex assembled from the standard chain complexes: \\
	\begin{center}
		$\gchains(\gsimplex^n)$, $\gchains(\gcube^n)$, ...
	\end{center}

	\pause
	\colorit{Our goals (loosely stated)} \\
	Understand the diagonal map of these standard complexes better to:

	\pause\smallskip
	\colorit{1)} Present effective/local computations of finer invariants in cohomology.

	\pause
	\colorit{2)} Describe explicit algebraic models of the homotopy type of spaces.
\end{frame}