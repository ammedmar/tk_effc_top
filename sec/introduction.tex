%!TEX root = ../effc_top.tex

\begin{frame}{Viewpoint}
	\pause
	\colorit{A goal of algebraic topology} \\
	To construct invariants of spaces up to some notion of equivalence.

	\bigskip\pause
	\colorit{Today} \\
	CW complexes and homotopy equivalence.

	\bigskip\pause
	\colorit{A basic tension} \\
	Computability \colorit{vs} strength of invariants.

	\bigskip\pause
	\colorit{Example} \\
	Cohomology \colorit{vs} homotopy.

	\bigskip\pause
	\colorit{A more subtle one} \\
	Effectiveness \colorit{vs} functoriality of their constructions.

	\bigskip\pause
	\colorit{Example} \\
	Cohomology via cochain complex \colorit{vs} maps to Eilenberg-Maclane spaces.
\end{frame}

\begin{frame}{Effectively defined cohomology}
	\pause
	\colorit{Poincar\'{e}'s idea} \\
	Break spaces into contractible combinatorial pieces: \\
	Simplices, cubes, ...

	\bigskip\pause
	\colorit{Kan--Quillen's idea} \\
	Replace spaces by functors with a geometric realization: \\
	Simplicial sets, cubical sets, ...

	\bigskip\pause
	\colorit{Cohomology} \\
	Cochain complex assembled from the standard objects: \\
	$\gchains^\bullet(\gsimplex^n)$, $\gchains^\bullet(\gcube^n)$, ...

	\bigskip\pause
	\colorit{Our goal (loosly stated)} \\
	Understand these pieces better to enhance cohomology with finer effectively computable invariants.
\end{frame}

\begin{frame}{Shortcomings of cohomology}
	\pause With mod 2 coefficients the real projective plane and the wedge of a sphere and a circle are isomorphic
	\[
	H^\bullet(\R \rP^2; \Ftwo) \cong H^\bullet(S^1 \vee S^2; \Ftwo)
	\]
	as graded vector spaces.

	\bigskip\pause
	Similarly,
	\[
	H^\bullet(\bC \rP^2; \Z) \cong H^\bullet(S^2 \vee S^4; \Z)
	\]
	as graded abelian groups.

	\bigskip\pause
	\colorit{Cup product} \\
	These can be distinguished by the algebra structure in $H^\bullet$.
\end{frame}