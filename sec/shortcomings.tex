%!TEX root = ../effc_top.tex

\begin{frame}{Shortcomings of cohomology I}
	\pause
	As graded vector spaces
	\[
	H^\bullet(\R \rP^2; \Ftwo) \cong H^\bullet(S^1 \vee S^2; \Ftwo).
	\]

	\pause
	Similarly, as graded abelian groups
	\[
	H^\bullet(\bC \rP^2; \Z) \cong H^\bullet(S^2 \vee S^4; \Z).
	\]

	\pause
	These can be distinguished by the \colorit{product structure} in $H^\bullet$.

	\bigskip\pause
	Defined by dualizing an \textbf{explicit} chain approximation to the diagonal
	\[
	\gchains(\gsimplex^n) \to \gchains(\gsimplex^n) \otimes \gchains(\gsimplex^n)
	\]
	due to Alexander and Whitney.

	\medskip\pause
	Similarly, Cartan and Serre constructed
	\[
	\gchains(\gcube^n) \to \gchains(\gcube^n) \otimes \gchains(\gcube^n).
	\]
\end{frame}

\begin{frame}[fragile]{Shortcomings of cohomology II}
	\pause
	Let $\Sigma$ denotes suspension, for example $\Sigma(S^1)$ is
	\begin{center}
		\includegraphics[scale=.25]{aux/suspension.pdf}
	\end{center}

	\pause\vskip-10pt
	As graded rings
	\[
	H^\bullet(\Sigma(\bC \rP^2)) \cong H^\bullet(\Sigma(S^2 \vee S^4)).
	\]

	\pause
	These can be distinguished by the action of the Steenrod algebra on $H^\bullet$.

	\bigskip\pause
	From the spectral viewpoint this structure is present by definition.

	\bigskip\pause
	\colorit{Question}: Can it be described \textbf{explicitly} at the chain level?
\end{frame}