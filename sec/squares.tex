%!TEX root = ../effc_top.tex

\begin{frame}[fragile]{Cup product}
	\pause Alexander and Whitney defined the cup product by dualizing a chain approximation to the diagonal:
	\[
	\gchains(\gsimplex^n) \to \gchains(\gsimplex^n) \otimes \gchains(\gsimplex^n).
	\]
	\pause Similarly, Cartan and Serre constructed: $\gchains(\gcube^n) \to \gchains(\gcube^n) \otimes \gchains(\gcube^n)$.

	\bigskip\pause
	As mentioned before, as graded rings,
	\[
	H^\bullet(\bC \rP^2) \not\cong H^\bullet(S^2 \vee S^4).
	\]

	\vskip -8pt \pause But,
	\[
	H^\bullet(\Sigma(\bC \rP^2)) \cong H^\bullet(\Sigma(S^2 \vee S^4)),
	\]
	where $\Sigma$ denotes suspension, for example $\Sigma(S^1)$ is
	\begin{center}
		\includegraphics[scale=.2]{aux/suspension.pdf}
	\end{center}
\end{frame}

\begin{frame}{Steenrod squares}
	\pause These chain approximations, unlike the diagonal of spaces, are \textcolor{pblue}{not} invariant under transposition: $x \otimes y \stackrel{T}{\mapsto} y \otimes x$.
	\begin{center}
		\begin{tikzpicture}
		\draw[color=pblue, thick] (0,0)--(1,1);
		\draw[->] (1.25, .5) -- (1.75, .5);
		\end{tikzpicture}
		\begin{tikzpicture}
		\node at (-0.1, 1){};
		\draw[color=pblue, thick] (0,0)--(1,0)--(1,1);
		\draw (1,1)--(0,1)--(0,0);
		\end{tikzpicture}
	\end{center}

	\smallskip\pause To correct homotopically the breaking of this symmetry, Steenrod introduced explicit maps
	\[
	\Delta_i \colon \gchains(\gsimplex^n) \to \gchains(\gsimplex^n)^{\otimes 2}
	\quad \text{satisfying} \quad
	\partial \Delta_{i} = \big(1 \pm T \big) \Delta_{i-1}
	\]

	\pause

	These induce further structure:
	\[
	\begin{split}
		Sq^k \colon H^\bullet(X; \Ftwo) &\to H^\bullet(X; \Ftwo) \\
		[\alpha] &\mapsto \big[ (\alpha \otimes \alpha) \Delta_i(-) \big]
	\end{split}
	\]

	\vskip-8pt\pause

	\textcolor{pblue}{Distinguishes}
	\[
	H^\bullet(\Sigma(\bC \rP^2)) \not\cong H^\bullet(\Sigma(S^2 \vee S^4))
	\]
\end{frame}

\begin{frame}[fragile]{A (new) description of Steenrod's construction}
	\pause\vskip-5pt
	\textcolor{pblue}{Notation:}
	\vspace*{-5pt}
	\[
	d_u[v_0, \dots, v_m] = [v_0, \dots, \widehat v_u, \dots, v_m]
	\]
	\pause\vspace*{-15pt}
	\[
	\rP_q^n = \set[\big]{U \subseteq \{0, \dots, n\} : \bars{U} = q}
	\]
	\pause\vspace*{-15pt}
	\[
	\forall \, U = \{u_1 < \dots < u_q\} \in \rP_q^n
	\]
	\pause\vspace*{-15pt}
	\[
	d_U = d_{u_1} \dotsm \, d_{u_q}
	\]
	\pause\vspace*{-15pt}
	\[
	U^\varepsilon = \big\{ u_i \in U \mid u_i + i \equiv \varepsilon \text{ mod } 2 \big\}
	\]

	\bigskip\pause
	\textcolor{pblue}{Definition (Med.)} \\
	For a basis element $x \in \gchains_m(\gsimplex^n)$
	\vspace*{-5pt}
	\[
	\Delta_i(x) \ = \!\!\! \sum_{U \in \rP_{m-i}^n} \!\! d_{U^0}(x) \otimes d_{U^1}(x)
	\]
	\vspace*{-10pt}

	\pause
	\textcolor{pblue}{Example:}
	\vspace*{-5pt}
	\begin{align*}
	\Delta_0 [0,1,2] &=
	\Big( d_{12} \otimes \id + d_2 \otimes d_0 + \id \otimes d_{01} \Big) [0,1,2]^{\otimes 2} \\ &=
	[0] \otimes [0,1,2] + [0,1] \otimes [1,2] + [0,1,2] \otimes [2].
	\end{align*}
\end{frame}

\begin{frame}{Fast computation of Steenrod squares}
	\pause
	Comparing with SAGE: (algorithm based on EZ-AW contraction)

	\smallskip\pause
	\textcolor{pblue}{$Sq^1$} on \textcolor{pblue}{$\Sigma^i\R P^2$} ($i^\th$ suspension of the real projective plane)
	\medskip
	\includegraphics[width=\textwidth]{aux/comp_sus_rp2.pdf}
\end{frame}

\begin{frame}[fragile]{Steenrod barcodes}
	\pause
	Given a filtered simplicial complex $X$
	\[
	X_0 \to X_1 \to \cdots \to X_n.
	\]

	\pause
	Cohomology induces a \textcolor{pblue}{persistent module}, its \textcolor{pblue}{barcode} is a summary of how Betti numbers are shared.
	\phantom{$Sq^k$ induces an endomorphism}
	\[
	\begin{tikzcd}[column sep = 15]
		H^\bullet(X_n; \Ftwo) \arrow[r] & \cdots \arrow[r] & H^\bullet(X_{n-1}; \Ftwo) \arrow[r] & \,H^\bullet(X_0; \Ftwo) \phantom{.}
	\end{tikzcd}
	\]
\end{frame}

\begin{frame}[fragile]{Steenrod barcodes}
	Given a filtered simplicial complex $X$
	\[
	X_0 \to X_1 \to \cdots \to X_n.
	\]
	Cohomology induces a \textcolor{pblue}{persistent module}, its \textcolor{pblue}{barcode} is a summary of how Betti numbers are shared.
	$Sq^k$ induces an endomorphism
	\[
	\begin{tikzcd}[column sep = 15]
	H^\bullet(X_n; \Ftwo) \arrow[r] & \cdots \arrow[r] & H^\bullet(X_{n-1}; \Ftwo) \arrow[r] & H^\bullet(X_0; \Ftwo) \\
	H^\bullet(X_n; \Ftwo) \arrow[u, "Sq^k"] \arrow[r] & \cdots \arrow[r] & H^\bullet(X_{n-1}; \Ftwo) \arrow[u, "Sq^k"] \arrow[r] & H^\bullet(X_0; \Ftwo) \arrow[u, "Sq^k"].
	\end{tikzcd}
	\]

	\pause
	The \textcolor{pblue}{$Sq^k$-barcode} of $X$ is the barcode of $\mathrm{img}\ Sq^k$.

	\bigskip\pause
	With \textit{Umberto Lupo} and \textit{Guillaume Tauzin} from \textcolor{pblue}{\texttt{giotto-tda}}'s team
	\medskip
	develop a high-performance implementation: \textcolor{pblue}{\texttt{steenroder}}.
\end{frame}

%\begin{frame}{Comparing persistent $Sq^2$-modules} \pause
%	Filtrations of the cone on the suspension of $S^2 \vee S^4$ and $\bC \rP^2$.
%
%	\pause
%	\begin{figure}
%		\centering
%		\begin{subfigure}[b]{0.49\textwidth}
%			\centering
%			\includegraphics[width=\textwidth]{aux/s2_s4.pdf}
%			\caption{$\mathrm C\,\Sigma(S^2 \vee S^4)$}
%			\label{f:s2_s4}
%		\end{subfigure}
%		\begin{subfigure}[b]{0.49\textwidth}
%			\centering
%			\includegraphics[width=\textwidth]{aux/cp2.pdf}
%			\caption{$\mathrm C\,\Sigma\,\bC\rP^2$}
%			\label{f:cp2}
%		\end{subfigure}
%	\end{figure}
%\end{frame}

\begin{frame}{Space of conformations of $\mathrm{C_8H_{16}}$}
	Points in $\R^{24}$ (positions of $8$ carbons in $\R^3$)

	\pause\smallskip
	Computing $Sq^1$ barcode of a ``smooth component'' of this point cloud
	\smallskip
	\includegraphics[width=\textwidth]{aux/cyclo-octane_subsampled_absolute_barcodes.pdf}
	Consistent with a \textcolor{pblue}{Klein bottle} component.
\end{frame}

\begin{frame}{More on cup-$i$ constructions}
	\pause
	\textcolor{pblue}{Theorem (Med.)} \\
	All cup-$i$ constructions in the literature are equal up isomorphism:
	\[
	\triangle \sim \triangle^\prime \iff \forall i \in \N, \ \triangle_i = \triangle_i^\prime \ \vee \, \triangle_i = T \triangle_i^\prime.
	\]
	(Proven via an axiomatic characterization.)

	\bigskip\pause
	\textcolor{pblue}{Theorem (Med.)} \\
	Steenrod's cup-$i$ construction defines the nerve of higher categories.

	\bigskip\pause
	\textcolor{pblue}{Theorem (Laplante-Anfossi--Med.--Vallette)} \\
	Let $P \subset \R^n$ be an $n$-dim convex polytope.
	A generic basis of $\R^n$ defines a cellular cup-$i$ construction $S^\infty \times P \to P \times P$.

	\bigskip\pause
	\textcolor{pblue}{Theorem (Cantero-Mor\'an)} \\
	Interpretation of these formulas for Khovanov homology.
\end{frame}

\begin{frame}{Relations}
	\pause
	There are two main Steenrod square relations:

	\bigskip\pause
	\textcolor{pblue}{Cartan}
	\vspace*{-5pt}
	\begin{equation*}
		Sq^k \big( [\alpha] [\beta] \big) =
		\sum_{i+j=k} \, Sq^i\big([\alpha]\big) Sq^j\big([\beta]\big),
	\end{equation*}

	\pause
	\textcolor{pblue}{Adem}
	\vspace*{-5pt}
	\begin{equation*}
		Sq^i Sq^j =
		\sum_{k=0}^{\lfloor i/2 \rfloor} \binom{j-k-1}{i-2k} Sq^{i+j-k} Sq^k.
	\end{equation*}

	\medskip\pause
	\textcolor{pblue}{Construction (Brumfiel--Med.--Morgan)} \\
	Explicit cochains witnessing these relations at the cochain level.

	\medskip\pause
	\textcolor{pblue}{Application (Gaiotto, Kapustin, Thorngren and others)} \\
	Used these in the classification of topological phases.

	\medskip\pause
	\textcolor{pblue}{Vague idea} \\
	Cochains as fields on triangulated spacetime with actions using these.
\end{frame}